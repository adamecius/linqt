\hypertarget{index_intro_sec}{}\section{Introduction}\label{index_intro_sec}
T\+B-\/\+Num\+Cal is a program aimed to perform different types of numerical calculations in tight-\/binding models. In the core of the program we use mainly the Kernel Polynomial method to compute different spectral quantities such as the conductivity tensor, the non-\/equilibrium spin-\/density or the density of states. Although a complementary approach, the Time-\/\+Evolution method is also implemented.

The program is designed to work using both M\+P\+I and Open\+M\+P paradigms of parallelism. Although the parallelism works different in each approach for the sake of performance. Instead of Open\+M\+P the program can benefit from the plataform C\+U\+D\+A for G\+P\+U calculations, which in many case result in a noticeable increasement in speed.\hypertarget{index_install_sec}{}\section{Installation}\label{index_install_sec}
The installation process is very simple, however, for optimal performance some tuning must be performed. For the moment, the program is entirely tested within Intel Parallel 2016, therefore the variables I\+N\+T\+E\+L\+\_\+\+H\+O\+M\+E, M\+P\+I\+\_\+\+H\+O\+M\+E, O\+M\+P\+\_\+\+H\+O\+M\+E and C\+U\+D\+A\+\_\+\+H\+O\+M\+E should be set in the arch\+\_\+make file. For now the I\+N\+T\+E\+L\+\_\+\+H\+O\+M\+E variable is mandatory, but if M\+P\+I\+\_\+\+H\+O\+M\+E, O\+M\+P\+\_\+\+H\+O\+M\+E or C\+U\+D\+A\+\_\+\+H\+O\+M\+E is not set, then the compilation will be performed excluding this options of parallelism, if both C\+U\+D\+A and O\+M\+P are set, C\+U\+D\+A takes priority over O\+M\+P. 