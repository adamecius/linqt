{\bfseries Introduction} Given an arbitrary state $|\Psi\rangle$, its evolution in time is given by the Schrodinger equation\+: \[ i\hbar \frac{d}{dt}|\Psi(t)\rangle = H |\Psi\rangle \] where $H$ the hamiltonian of the system, $\hbar$ the Planck\textquotesingle{}s constant, and $|\Psi(t)\rangle$ the state of the system at a given time $t$. For time-\/independent Hamiltonians, the time-\/evolve state has a simple expression\+:

\[ |\Psi(t)\rangle = U(t)|\Psi\rangle \]

where $U(t)\equiv{\rm e}^{i H t}$ an unitary operator known as time-\/evolution operator. Typically, the calculation of the exponential of an operator is not a trivial task. In Lin\+QT we circumvent this problem by expanding it in terms of Chebyshev polynomials. Following the approach in \href{https://journals.aps.org/rmp/abstract/10.1103/RevModPhys.78.275}{\tt 1}, we first express the evolution operator in terms of the normalized hamiltonian $\tilde{H}$ \[ U(t)={\rm e}^{i \omega_c t} {\rm e}^{i \omega \tilde{H}}\ \] where $\omega_c\equiv E_c/\hbar$ with $E_c$ the band center, and $\omega\equiv\frac{W}{2\hbar}$ with $W$ the bandwidth.

\[ U(t) = \sum_{m=0}^{\infty} U_m(t) \] 