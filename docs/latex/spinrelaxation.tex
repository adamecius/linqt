{\bfseries Introduction} Systems described by a spin-\/dependent Hamiltonians, endows spins with dynamics. Typically, crystalline systems will posses spin-\/dependent fields which will make the spins to precess coherently. This is the case of external magnetic fields and spin-\/orbit coupling fields. However, in the presence of randomness, the spin-\/coherence is lost due to the irreversibility, making it relax over some relevant time scale $\tau_{\rm s}$ know as spin relaxation time.

To obtain the spin relaxation time it is sufficient to solve the Schrodinger equation\+: \[ i\hbar \frac{d}{dt}|\Psi(t)\rangle = H |\Psi\rangle \] where $H$ the hamiltonian of the system, $\hbar$ the Planck\textquotesingle{}s constant, and $|\Psi(t)\rangle$ the state of the system. Then, spin density is computed in the Heisenberg picture as \[ S( t) = {\rm Tr}\left[ \rho\frac{s(t)}{\Omega} \right], \] where \$\$ the density matrix, $s(t)=U^\dagger(t) s U(t)$ the time evolved spin operator, with $s$ its static representation in the Schrodinger picture and $U(t)$ the time-\/evolution operator.

In Lin\+QT we deal with tight-\/binding models of solids described by a time-\/independent Hamiltonian. Therefore, the evolution operator can be written simply as $U(t) = {\rm e}^{-i H t/\hbar}$, and the density matrix depends on the Fermi energy $\varepsilon_{\rm F}$. To compute the evolution of the spin density, we follow the approach done by \mbox{[}Cummings et. al\mbox{]}\mbox{[}1\mbox{]}, in which the density matrix takes the following form

\[ \rho(\varepsilon_{\rm F}) = \frac{ P \delta(H-\varepsilon_{\rm F}) P }{ {\rm Tr}[\delta(H-\varepsilon_{\rm F})] } \]

with $ P $ a projector operator. In \mbox{[}1\mbox{]}, the projector operator is choosen such that spin density acts on spin polarizes in a given spatial dimension. If such spatial direction is described by an altitude angle $\theta$ and a azimutal angle $\phi$, then the spinorial component of the projector operator takes the form of\+: \[ P_{\pm}(\theta,\phi) =\frac{1}{2} \left(\begin{array}{cc} 1\pm \cos(\theta) & {\rm e}^{-i\phi}\sin\theta \\ {\rm e}^{i\phi}\sin\theta &1\mp \cos(\theta) \end{array} \right) \] but an arbitrary operator can be choose.

{\bfseries The Chebyshev approach}

We rewrite the expression by using the permutation property of the trace, \[ S( t) = \frac{ {\rm Tr}\left[ PU^\dagger(t) sU(t) P\delta(H-\varepsilon_{\rm F})\right]}{Z}, \] where we had defined $Z={\rm Tr}\left[ P\delta(H-\varepsilon_{\rm F})\right]$. The trace is then approximate by a mean over a random phase vector \[ \mu_{m,n} = \frac{1}{Z}\langle \chi_{\rm L}(n)| s|\chi_{\rm R}(n,m)\rangle, \] where $ |\chi_{\rm L}(n) \rangle = U(t_n)P|\chi\rangle $ and $ |\chi_{\rm R}(m,n) \rangle = U(t_n)PT_m(H)|\chi\rangle $

Finally, the \hyperlink{time_evolution}{The time evolution} is computed in terms of the expansion moments as\+: \[ S(t_n)= \sum_{m}\mu_{m,n} T_m(\varepsilon_{\rm F}) \]

\mbox{[}1\mbox{]}\+: \href{https://journals.aps.org/prl/abstract/10.1103/PhysRevLett.119.206601}{\tt https\+://journals.\+aps.\+org/prl/abstract/10.\+1103/\+Phys\+Rev\+Lett.\+119.\+206601} \hypertarget{time_evolution}{}\section{The time evolution}\label{time_evolution}
{\bfseries Introduction} Given an arbitrary state $|\Psi\rangle$, its evolution in time is given by the Schrodinger equation\+: \[ i\hbar \frac{d}{dt}|\Psi(t)\rangle = H |\Psi\rangle \] where $H$ the hamiltonian of the system, $\hbar$ the Planck\textquotesingle{}s constant, and $|\Psi(t)\rangle$ the state of the system at a given time $t$. For time-\/independent Hamiltonians, the time-\/evolve state has a simple expression\+:

\[ |\Psi(t)\rangle = U(t)|\Psi\rangle \]

where $U(t)\equiv{\rm e}^{-i H t}$ an unitary operator known as time-\/evolution operator. Typically, the calculation of the exponential of an operator is not a trivial task. In Lin\+QT we circumvent this problem by expanding it in terms of Chebyshev polynomials. Following the approach in \href{https://journals.aps.org/rmp/abstract/10.1103/RevModPhys.78.275}{\tt 1}, we first express the evolution operator in terms of the normalized hamiltonian $\tilde{H}$ \[ U(t)={\rm e}^{-i \omega_c t} {\rm e}^{-i \omega \tilde{H}}\ \] where $\omega_c\equiv E_c/\hbar$ with $E_c$ the band center, and $\omega\equiv\frac{W}{2\hbar}$ with $W$ the bandwidth.

\[ U(t) = \sum_{m=0}^{\infty} U_m(t) T_m(\tilde{H}) \] where \[ U_m(t) = (-i)^m(2-\delta_{m0})J_m(\omega t){\rm e}^{-i \omega_c t} \] 